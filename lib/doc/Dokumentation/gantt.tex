
\newenvironment{Wochentabelle}{
\subsection{Arbeit der \homeworkProblemName}
\noindent
\begin{figure}[H]
\begin{center}
\begin{ganttchart}[y unit title=0.4cm,
y unit chart=0.5cm,
x unit=0.3cm,
vgrid,%hgrid, 
title label anchor/.style={below=-1.6ex},
title left shift=.05,
title right shift=-.05,
title height=1,
bar/.style={fill=gray!50},
incomplete/.style={fill=white},
progress label text={},
bar height=0.7,
group right shift=0,
group top shift=.6,
group height=.3,
group peaks height=.2]{1}{40}
%labels
\gantttitle{\homeworkProblemName}{40} \\
\gantttitle{Montag}{8} 
\gantttitle{Dienstag}{8} 
\gantttitle{Mittwoch}{8} 
\gantttitle{Donnerstag}{8} 
\gantttitle{Freitag}{8} \\
}{
\end{ganttchart}
\end{center}
\caption{Wochenplan \arabic{Wochencounter}}
\end{figure}
}