\section{Introduction}
%introduction of the topic
Algorithms are everywhere. They are a part of our everyday lives without even realizing it. They can be found in general mathematics, or in practical applications such as  computers or smartphones. But what exactly is an algorithm? According to the Oxford Dictionary an algorithm is generally speaken "a process or set of rules to be followed in calculations or other problem-solving operations, especially by a computer" \cite{web:oxforddictionaries2018}.
In our apprenticeship, we both have a lot to do with algorithms. Damian works as a software engineer, where he writes most of the time software. Stefan is an electronics engineer and his specialisation in the fourth year of education is also software. According to that, we both have a pretty good idea of what algorithms are and what they do. We found out, that each of us is fascinated by the concepts and possibilities algorithmic structures provide. So we came up with the idea, to implement an algorithm by ourselves. \bigskip

%goals
We set the goal, to program an algorithm, that can play the game Ultimate Tic-Tac-Toe (UTTT). UTTT is a more complex and strategic version of the ordinary Tic-Tac-Toe. 
Our approach to solve that problem is a so called Self Learning Alpha Beta Pruning Algorithm, also known as SLAP Algorithm. It's a combination of the already existing Alpha Beta Pruning algorithm combined with an own made up extension which is, as the name suggests, self learning. We want to analyse the learning process of the algorithm in more detail and evaluate whether there is really an improvement. This will be the mathematical part of our work
To experience our algorithm in action, we also wanted to create a website where everybody can play the game against our SLAP Algorithm.
Another intention of our project is to bring the seemingly complex subject in an easy understandable form to the interested reader. We aim to resume our approach of the solution in a comprehensible way. We want to give the reader an insight and a deeper understanding of algorithms, how they work and how they are implemented. With the website, we hope to create an interesting and interactive extension to this paper.
In order to meet the requirements to cover two subjects, we decided to write our paper and the website in English. \vspace{8mm}

%overview methodological approach
The first part of our IDPA will be to implement the game itself and the corresponding SLAP algorithm. 
That includes to find a way to value the states of the game in the most efficient way. This will be the task of the evolutionary algorithm. Only if that part of the SLAP algorithm works fine, the Alpha Beta Pruning is able to work in our favour.
We decided to realise all of that with a modern programming language called Dart from Google. Dart is a well-structured, object-oriented programming language that gives us the flexibility to write client and server-side code. All with one code base and one language. Once everything is set up, we will extend our project, that we are able to track the development of the self learning part. We are also planning to integrate a function into the website where the user can train his own SLAP algorithm and observe the progress themselves.
%overview structure of the paper
%to do

\section{Rules}
To understand to game Ultimate Tic-Tac-Toe, it is required to  know the rules of the ordinary Tic-Tac-Toe game. If you already know the rules of the ordinaty Tic-Tac-Toe, you can continue reading paragraph 2.2. 

\subsection {Rules Tic-Tac-Toe}
The game Tic-Tac-Toe consist of a 3-by-3 board. Two players are required while one player represents X and the the other player represents O. One player can start and put his mark anywhere he'd like to. After that, the other player can take his turn and put his mark to a remaining spot. This procedure continues until someone has 3 of his own marks vertically, horizontally or diagonal aligned. It's possible that no party wins and the match ends undecided.

\begin{fixedpic}
	\centering
	\includegraphics[width=0.2\textwidth]{tictactoe}
	\captionof{figure}{a random scenario of an ordinary Tic-Tac-Toe field}
\end{fixedpic}


\subsection {Rules Ultimate Tic-Tac-Toe}
The board of Ultimate Tic Tac Toe is made up from 9 ordinary Tic-Tac-Toe boards, hence there are 81 possible fields.
Each small Tic Tac Toe board will be called a local board and the big Tic-Tac-Toe board will be called the global board.
A player can start anywhere he'd like to on a local board. According to the location he played on a local board, his opponent will be sent to that position in the global board. The opponent can now play on the local board, keeping in mind that he will send the other player to the relative position on the global board. 

A victory of a local board is the equivalent to a marked tile on the global board. To win the game, one has to win 3 horizontally, vertically or diagonal aligned tiles of the global board.

If a local board is won or draw, no more moves are allowed there. In case a player was sent to such a board, he is allowed to make his next turn on every other local board. It's possible that the game ends in a draw, because  no more legal moves are allowed.

\begin{fixedpic}
	\centering
	\includegraphics[width=0.6\textwidth]{uttt}
	\captionof{figure}{a random scenario of an Ultimate Tic-Tac-Toe field}
\end{fixedpic}

The yellow marked local fields are already won by the yellow party.

\section{SLAP Algorithm}

\subsection{Alpha Beta}
\begin{fixedpic}
	\centering
	\includegraphics[width=0.9\textwidth]{alphabetatree}
	\captionof{figure}{a representation of a Alpha Beta Pruning search tree}
\end{fixedpic}



\todo{erster satz nicht ganz klar}
Our algorithm takes a score. Each score has a certain number of moves that can be played. All the possible moves are now tried out one after the other. After each move we get a new score, which also has possible moves again. Even after these moves, we have new scores again. This is how one score after the other is calculated. This can be represented as a tree structure. Each knot (whether round or square) symbolizes a score. The branches show the possible moves. Since evaluating all scores would be too computationally intensive, a search depth is defined. In the figure the search depth would be 4, because four moves were evaluated from the main node. Now the scores of the lowest row (hands) are evaluated. The score evaluation in the graph is represented by the number in knots (do not get confused by the numbers in the other knots, more on this later). The bigger the number, the better the score for us. Now the tree is evaluated so that in the worst case the score is as high as possible for us. That means as much as assuming that the opponent always makes the move that would be worst for us. This is how the evaluation starts below. The fourth move (branches between the last two levels) is played by the opponent (MIN stands for minimizing player). Of course, the opponent takes the worst score for us. Therefore, the smallest number of leaves is always written in the upper node. See the example at the bottom left:
\begin{fixedpic}
	\centering
	\includegraphics[width=0.15\textwidth]{alphabetabranch}
	\captionof{figure}{the smaller value gets written to the upper knot}
\end{fixedpic}
The opponent has two moves from this score. Either he takes the first move, which leads to a score of 5, or he takes the second move, which leads to a score of 6. The first move is clearly an advantage for the opponent. Therefore, the parent node is given the value five. This happens with all nodes of this level. The third move is now played by us. We'll take the highest possible score, of course. In this way, the higher-level nodes of the previously filled level are filled with the highest of the possible values. Now it's the turn of the opponent who chooses the worst score for us. Thus, the procedure repeats itself up to the top node. In this way we get the information which move is best for us. The grey marked nodes and crossed out branches are optimizations of the algorithm. So we do not have to evaluate these areas at all and can save computing time.

\begin{fixedpic}
	\centering
	\includegraphics[width=0.9\textwidth]{alphabetalevel}
	\captionof{figure}{all possible game moves in the 3rd level of the search depth, evaluated by the minimzing player}
\end{fixedpic}

The average amount of options per move in the game Ultimate Tic-Tac-Toe is five. \cite{web:tsurel2013}
In our case, the Alpha Beta Pruning search tree can be pictured with five branches emerging from every new node. With a search depth of n, the calculated amounts of states of all nodes  can be calculated with the following  equation:  $$\alpha = \sum_{i=1}^{n} 	\delta^{i}$$

\begin{center}
$\alpha$: amount of nodes,
$n$: search depth,
$i$: index,
$\delta$: average move per node
\end{center}


\subsection{Heuristic}
The Alpha Beta Pruning only decides which move is played. But without the heuristic, the Alpha Beta Pruning algorithm would be pointless. The task of the heuristic is to fill the nodes in the Alpha Beta Pruning search tree with meaningful numbers. The more accurate our heuristic will be, the better are the chances for our algorithm to win. The heuristic has only the purpose to evaluate the score of the board, or in other words, calculate a number for each possible game state. The better the game state, the higher the score.

Now to how the evaluation works. We award points for certain conditions. There are exactly five of these conditions. The check always takes place in a row, a column or a diagonal. The first two conditions are checked in the small fields.
\begin{itemize}
\item One field occupied, two fields empty. The condition occurs three times here. Once in the right column, once in the bottom row and once in the diagonal from top left to bottom right.
\begin{fixedpic}
	\includegraphics[width=0.2\textwidth]{con1}
	\captionof{figure}{Condition one}
\end{fixedpic}
\item Two fields occupied, one field empty. Here the condition occurs in the lowest row. Note how the first condition occurs three times again, in the left column vertically, diagonally left bottom to the right bottom and again vertically middle column.
\begin{fixedpic}
	\includegraphics[width=0.2\textwidth]{con2}
	\captionof{figure}{Condition two}
\end{fixedpic}
\end{itemize}
The following three conditions are checked in the large field.
\begin{itemize}
\item One small field won, two small fields not yet determined. Here the condition occurs once in the top row. In this figure, the first two conditions are no longer considered.
\begin{fixedpic}
	\includegraphics[width=0.2\textwidth]{con3}
	\captionof{figure}{Condition three}
\end{fixedpic}
\item Two small fields won, one small field not yet determined. Here is a visualization of this condition.
\begin{fixedpic}
	\centering
	\includegraphics[width=0.6\textwidth]{con4}
	\captionof{figure}{Condition four}
\end{fixedpic}
\item Three small squares won. This also corresponds to a victory. Here the condition occurs in the right column.
\begin{fixedpic}
	\centering
	\includegraphics[width=0.6\textwidth]{con5}
	\captionof{figure}{Condition five}
\end{fixedpic}
\end{itemize}
In the following figure, all five occurring states are marked with different colors:
\begin{description}
\item[State one] Yellow
\item[State two] Green
\item[State three] Blue
\item[State four] Purple
\item[State five] Light blue
\end{description}
\begin{fixedpic}
	\centering
	\includegraphics[width=0.6\textwidth]{allcons}
	\captionof{figure}{All Conditions}
\end{fixedpic}
The states are determined separately for both players. We take the difference of the states of the two players. 
\todo{evtl erklären, wieso wir die differen der beiden spieler nehmen}
Then we have a certain number for each state. We have to convert them into a meaningful number. This is where the DNA comes into play. The DNA is an object consisting of 5 fields. Each field stands for a factor by which the number of states is multiplied. Not all parameters should contribute the same value to the game state evaluation. With the DNA, we are able to control the relative proportions of the five parameters. Let's do a sample calculation using the picture above. Our DNA has the following Numbers.\\ 
\begin{description}
\item[Factor state one] 1
\item[Factor state two] 3
\item[Factor state three] 10
\item[Factor state four] 30
\item[Factor state five] 100
\end{description}

In other words, the 5th parameter contributes 100 times more value to the end evaluation than the 1st parameter. The 4th parameter is 10 times more valuable than the 2nd parameter. In the following table, the whole calculation of the value of the game state is shown.

\begin{tabularx}{\textwidth}{|X|X|X|X|X|X|}
\hline
State & Occurrence red & Occurrence blue & Difference & Factor of DNA & Value \\\hline
1	& 0	& 11	& -11	& 1 	& -11 \\\hline
2	& 1	& 3 	& -2	& 3 	& -6 \\\hline
3	& 4	& 0 	& 4 	& 10	& 40 \\\hline
4	& 2	& 0 	& 2 	& 30	& 60 \\\hline
5	& 1	& 0 	& 1 	& 100	& 100 \\\hline
\multicolumn{5}{X|}{} & 183 \\\cline{6-6}
\end{tabularx}\\
\\
The effective score in this case is 183, which is all the heuristic does. 
Based on this information, the Alpha Beta Pruning decides which move to play.



\subsection{Self Learning}
To say it in easy terms, the self learning part learns how to estimate the game state in the most meaningful way. It learns which heuristic parameters are important in relation to the other heuristic parameters. In our case, the DNA  is responsible for that. The DNA dictates, which heuristic parameters are more important and it does that simply with a factor. The DNA is also the only thing which changes, while learning. 

Let's take a look at how an algorithm is generated and how it gets better over time.
To start a new era, we need to set a few things up. An era can be compared to several generations of humans. 
The era contains all generations. In a human generation, we have several individuals. In our case, we will call our individuals of the generation Organisms. We have to decide, how many Organisms per generation we want to create. But we have to keep in mind, the more Organisms we create, the longer the evolution will take. That's because every Organisms will play against every other Organisms twice. They play two times, to make sure every Organisms has once the possibility to start. The process of playing against each other is called selection. The Organisms get rewarded with 3 points when they win, and get 1 point when they end in a draw. When they loose, they won't get any points. The whole process of finding the best Organisms of a generation is actually very similar to a football cup.
%Formel 2*(n-1)
We also need to decide the search depth of the alpha beta pruning algorithm the organisms use, while playing against each other. The bigger the search depth, the longer it will take, because the computer has to calculate more scenarios. But the deeper we search, the more accurate results we will get. 
Now after we set up the amount of organisms and the search depth, we can start our new era. The first generation gets created (initialised) with random DNA. %in welchem bereich sind die Zufallszahlen?
That means the proportion of the different parameters vary vividly. We already know, the Organisms with the best random parameter configuration will perform the best. But we don't know yet exactly, which parameters are the most important ones. To find it out, which parameters are the best ones, we now let the organisms play against each other. In other words, we will start the selection. After all matches are finished, we can rank the organisms according to the point distribution system, which we used. The worse half of our first generation will die. And as it is in real evolution, mutation will take place.
The DNA of the better half of the ranking will mutate two times and form a new generation. The best Organism of the generation will survive and continue to live in the next generation.
%eventuell genäuere/bessere beschreibung
As described at the beginning, the mutation only affects the DNA. The DNA contains the factor for each parameter and each factor will be mutated by a value between 0.8 and 1.2. This results in an almost totally new set of Organisms, which are descenders of the previous generation. The selection starts again for this generation.
This cycle will continue as long as the creator of the era desires to. \\

\begin{fixedpic}
	\centering
	\includegraphics[width=0.6\textwidth]{evoalgo}
	\captionof{figure}{visualisation of an evolutionary algorithm}
\end{fixedpic}


The era which is shown on our website contains 70 generations and each generation consists of 16 Organisms.
%wie lange zum trainieren?
\todo{Wie lange zum trainiere? 8h an Computer mit 3.2Ghz und 16 Threads}

\subsection{Implementation in Dart}

%For dynamic parts of a website, webbrowsers such as Google Chrome or Windows Explorer only understand a language called JavaScript. Because of that, our project, %which is mainly written in Dart, gets compiled (translated) to JavaScript. In that way, we can avoid the disadvantages JavaScript has.

\subsection{Website and User Interface}
In order that our project is accessible all over the world with different end devices ranging from Smartphones to Personal Computers running with different Operating Systems, we thought that it would be the most convenient solution to make a website. 

In general, the static part of the website is made with HTML (Hyper Text Markup Language) and CSS (Cascaded Style Sheets). HTML specifies the structure of a website.  CSS is then required to style everything. The dynamic part was already discussed in the previous subsection. 
Because CSS can be extremly time consuming, we made use of a CSS Framework called Materialize from Google, which provides cross browser compatible and responsible components such as navbars, buttons, dropdown-lists, cards, modals and many more. To make our website more appealing and fluid, we used CSS animations. For this purpose, we used the website Animista, which provides ready-to-use CSS animations.

To deploy our website online, we also needed to have a hosting service which offers a server and a domain. Because we already worked with tools of Google like the language Dart and the Materialize Framework, it wasn't far-fetched, that we decided to make use of a hosting provider which also belongs to Google. It's called Firebase. Firebase provides many options for developers such as Real Time Databases or Cloud Messaging, but we only needed to use the hosting opportunity, which supports hosting static files like CSS and HTML.\\

Our website consists of two main parts. The first part, we called it "Reduced View", is mainly designed for an easy game experience, where you can choose between three levels to play against the algorithm. The levels include "easy", "medium" and "hard". Behind the easy level plays the best organism of the first generation. 
The medium level is played by the best organism of the 35th %evtl korrektur
generation and the hard level is the best organism of the whole era. The default search depth of the current game is three. Three is an experiential value, which is the best compromise between search duration and reliability of the algorithm.
The first part of the website contains an explanation of the rules of the game as well.

Via a button in the navbar, it's possible to change to a more detailed  part of the website, we called it "Advanced View", which we designed for people who are interested in what's going on behind the scenes. It's possible to see the whole era with all generations and organisms and it's evident, how the algorithm has evolved. It's also possible to generate a new era and calculate everything from scratch. As one plays against a desired Organism, it's also possible to adjust the search depth of the current game dynamically. It's observable, that it takes much longer for the algorithm to make a turn if the search depth is deeper. %evtl zeit erwähnen wie lange es geht

%evtl weglassen oder verbessern
Our Algorithm is separated from the part of the Web, that it is modular and reusable for any other applications like an app or something similar. In programming terms, things are separated with a thing called "Interface". Our Interface in dart is called the Player Interfacer. The Player Interface simply provides the information of our game state and is able to get input from the user interface on our website.

In HTML and CSS, we prepared a 9x9 grid, which we had to fill dynamically with the information of our Player Interface. When the algorithm made his turn, we also implementet a function that shows the player, where he has to play his next turn. 


\subsection{ Runtime } 

\subsubsection{ Device Restrictions}


\section{Progress Evaluation}
As already mentioned, first-generation organisms are produced from randomly generated DNA. Thus, on average, all parameters are still evaluated in approximately the same way. However, this changes over the generations.

\subsection{Expected Result}
According to instinct we can say that condition 5, a victory of the game, should be rated better than condition 1. We can also say that condition 2 should be rated better than condition 1, because with condition 2 we are closer to winning a local game. Also a won local game, condition 3, is better than a nearly won local game, condition 2.
The point is that the conditions are already sorted the way we expect them to be weighted. Condition 1 should have the least weight, condition 5 the most.

\subsection{Received results}
The first generation consists only of randomly created organisms. Nevertheless, a first development can already be observed. The best organism already shows more or less the expected curve. The following states are better evaluated than the previous ones. This probably also helped him to victory. Also the last organism is the opposite. With the exception of the last parameter, the following states are rated worse than the previous ones.
\begin{fixedpic}
\begin{dnadiagram}
\addplot coordinates{(1, 108) (2, 379) (3, 856) (4, 1648) (5, 1745)};
\addplot coordinates{(1, 1408) (2, 1575) (3, 1801) (4, 205) (5, 1508)};
\addplot coordinates{(1, 1586) (2, 790) (3, 494) (4, 439) (5, 697)};
\legend{First of Genration 1,Median of Generation 1,Last of Generation 1}
\end{dnadiagram}
\captionof{figure}{The best, an average and the last DNA of Generation 1}
\end{fixedpic}
While in the first generation only the first organism more or less reproduced the expected curve, in the 5th generation there are already 11 of 16 organisms with such DNA. The remaining 5 organisms show the curve up to the fourth parameter, but the fifth parameter was mutated badly and the curve goes down again. So we have in only five generations already a first trend which plays itself in and already strongly differs from the generation 1.
\begin{fixedpic}
\begin{dnadiagram}
\addplot coordinates{(1, 112) (2, 398) (3, 735) (4, 1410) (5, 1890)};
\addplot coordinates{(1, 122) (2, 362) (3, 750) (4, 1280) (5, 1646)};
\addplot coordinates{(1, 118)(2, 423)(3, 1078)(4, 1507)(5, 1123)};
\legend{First of Genration 5,Median of Generation 5,Last of Generation 5}
\end{dnadiagram}
\captionof{figure}{The best, an average and the last DNA of Generation 5}
\end{fixedpic}
In the tenth generation there is again a conspicuity. Slowly all organisms start to look the same, but places 3 and 4 stand out. In these organisms, the fourth and fifth parameters are almost identical. Probably because the fourth parameter mutated strongly upwards. Although these organisms do not look like the expected result, they are better than many that look like the expected result. The striking difference to the others is also that the relative difference of conditions 3 and 4 is greater than for the others. Precisely because the fourth parameter has probably mutated strongly upwards. According to this observation, the ratio of the parameters plays an important role.

\begin{fixedpic}
\begin{dnadiagram}
\addplot coordinates{(1, 99) (2, 555) (3, 730) (4, 1294) (5, 2055)};
\addplot coordinates{(1, 60) (2, 324) (3, 642) (4, 1664) (5, 1763)};
\addplot coordinates{(1, 92) (2, 281) (3, 796) (4, 2018) (5, 2055)};
\legend{First of Genration 10,Third of Generation 10,Fourth of Generation 10}
\end{dnadiagram}
\captionof{figure}{Some conspicuous DNA of Generation 10}
\end{fixedpic}

From the tenth generation onwards, these conditions are further developed. So the best of the tenth generation reminds rather of a line, whereas the best of the generation 20 reminds rather of a (half-)parabola. This development continues until generation 70, where we have completed the development.

\begin{fixedpic}
\begin{dnadiagram}
\addplot coordinates{(1, 99) (2, 555) (3, 730) (4, 1294) (5, 2055)};
\addplot coordinates{(1, 103) (2, 369) (3, 1110) (4, 2209) (5, 3033)};
\addplot coordinates{(1 ,74) (2 ,275) (3 ,732) (4 ,1387) (5 ,3976)};
\addplot coordinates{(1, 66) (2, 282) (3, 583) (4, 1451) (5, 5091)};
\addplot coordinates{(1, 61) (2, 327) (3, 717) (4, 1542) (5, 5683)};
\addplot coordinates{(1, 71) (2, 368) (3, 476) (4, 1800) (5, 8907)};
\legend{First of Generation 10,F.o.G. 15,F.o.G. 20, F.o.G. 30, F.o.G. 50, F.o.G. 70}
\end{dnadiagram}
\captionof{figure}{First}
\end{fixedpic}

\todo{Nach State und Condition suchen und alles aufräumen}

\section{Motivation}
Since the beginning of searching a topic for our IDPA, we knew that we both were interested in a practical project rather than a  project work in written form. We both are interested in coding and have to work with software. Because of that, a software project was the most obvious option. Additionally, to create an own software project was for both of us a desirable idea.
Damian came up with the idea, to create a Self Learning Algorithm which is able to play Ultimate Tic-Tac-Toe. We both were really interested in a practical implementation of such an algorithm. We both knew that we could learn a lot with such a project and that the context of the IDPA provides the perfect opportunity for that. \\

We also knew, that we could integrate knowledge in our project, which we gained at our mathematics teaching during the last three years. To evaluate and describe certain behaviours of our algorithm, algebra and data analysis provide perfect tools.

To cover two subjects in our IDPA, we decided to apply our English knowledge, which we also improved during the last years at the BMS. Because we write all our software in English, we came up with the idea to keep language consistency throughout our whole project, including the website and this documentation.

\subsection{Why we chose Dart}



\section{Challenges}


\section{Review}
In retrospect, it was a very interesting project to create. It was very motivating to us, to see the project grow while achieving small goals. Despite the fact that the algorithm works in theory, we had our concerns that it would seamlessly work in a practical implementation. It was a relieve, when we both saw, that the self learning part of the algorithm evolved in accordance to our expectations.
