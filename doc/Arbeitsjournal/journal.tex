\begin{dayentry}{\DTMdisplaydate{2018}{8}{23}{-1}}
	\task{Moser}{Created a \href{https://github.com/M0SERD/uttt/}{GitHub Repo}. Created the models for our program. In addition the skeleton of the GameStateController and AlphaBetaPruning-Algorithm.}
\end{dayentry}

\begin{dayentry}{\DTMdisplaydate{2018}{8}{29}{-1}}
	\task{Moser}{Added a template for the documentation.}
\end{dayentry}

\begin{dayentry}{\DTMdisplaydate{2018}{9}{10}{-1}}
	\task{Moser}{Added a template for the project scetch. Reorganized the documentation folder.}
\end{dayentry}

\begin{dayentry}{\DTMdisplaydate{2018}{9}{11}{-1}}
	\task{Jampen}{Added a hypothesis to the project scetch.}
\end{dayentry}

\begin{dayentry}{\DTMdisplaydate{2018}{9}{11}{-1}}
	\task{Moser}{Changed section \enquote{starting position} from project scetch.}
\end{dayentry}

\begin{dayentry}{\DTMdisplaydate{2018}{10}{9}{-1}}
	\task{Moser}{Translated project scetch to english.}
\end{dayentry}

\begin{dayentry}{\DTMdisplaydate{2018}{10}{10}{-1}}
	\task{Moser}{Changed title page of the documentation. Cleaned up Git repository and added entry to .gitignore file. Added first classes to create GridCache.}
	\task{Jampen}{Added some sections to the project scetch.}
\end{dayentry}

\begin{dayentry}{\DTMdisplaydate{2018}{10}{21}{-1}}
	\task{Moser}{First version, where computers can play against each other. Implemented GridCache, RandomAlgorithm and Computer class.}
	\task{Jampen}{Cleaned up project scetch.}
\end{dayentry}

\begin{dayentry}{\DTMdisplaydate{2018}{10}{22}{-1}}
	\task{Moser}{Removed unnecessary content in project scetch and documentation. Removed some duplications in project scetch. Corrected german side numberation.}
\end{dayentry}

\begin{dayentry}{\DTMdisplaydate{2018}{10}{26}{-1}}
	\task{Moser}{Changed title on the title page and changed some things in the project scetch.}
	\task{Jampen}{Refinement of project scetch.}
\end{dayentry}

\begin{dayentry}{\DTMdisplaydate{2018}{10}{29}{-1}}
	\task{Moser}{Added empty project for a website running with \href{https://dartlang.org}{Dart}. Refactored some folders in the Git repo.}
\end{dayentry}

\begin{dayentry}{\DTMdisplaydate{2018}{10}{30}{-1}}
	\task{Moser}{Refactored some folders in the Git repo, added support for APA citisation in the documents.}
\end{dayentry}

\begin{dayentry}{\DTMdisplaydate{2018}{10}{31}{-1}}
	\task{Moser}{Added code documentation and changed some project settings.}
\end{dayentry}

\begin{dayentry}{\DTMdisplaydate{2018}{11}{2}{-1}}
	\task{Moser}{Implemented the HeristicCache, AlphaBetaPruning algorithm, Evolution model, first lines of EvolutionController and some tests.}
	\task{Jampen}{Implemented the WebPlayer and Grid on the website. Now we have a first playable version against a random Algorithm.}
\end{dayentry}

\begin{dayentry}{\DTMdisplaydate{2018}{11}{5}{-1}}
	\task{Moser}{Fixed the AlphaBetaPruning Algorithm and added it to the website.}
\end{dayentry}

\begin{dayentry}{\DTMdisplaydate{2018}{11}{7}{-1}}
	\task{Moser}{Implemented evolutionary learning.}
	\task{Jampen}{Added blinking to the yellow background.}
\end{dayentry}

\begin{dayentry}{\DTMdisplaydate{2018}{11}{9}{-1}}
	\task{Moser}{Little optimisation of the evolution Object and Controller. Added results of the first trained DNA objects. First sucessful test of a Javascript worker.}
\end{dayentry}

\begin{dayentry}{\DTMdisplaydate{2018}{11}{13}{-1}}
	\task{Moser}{Create working journal}
\end{dayentry}

\begin{dayentry}{\DTMdisplaydate{2018}{11}{14}{-1}}
	\task{Moser}{Add some animations to the website.}
\end{dayentry}

\begin{dayentry}{\DTMdisplaydate{2018}{11}{18}{-1}}
	\task{Moser}{Implement first worker to simulate a player.}
\end{dayentry}

\begin{dayentry}{\DTMdisplaydate{2018}{11}{20}{-1}}
	\task{Moser}{Renamed some files in the Github repository.}
\end{dayentry}

\begin{dayentry}{\DTMdisplaydate{2018}{11}{21}{-1}}
	\task{Moser}{Made the website responsive and improved some things in the worker}
\end{dayentry}

\begin{dayentry}{\DTMdisplaydate{2018}{11}{22}{-1}}
	\task{Moser}{Finally a building app with functional worker.}
\end{dayentry}

\begin{dayentry}{\DTMdisplaydate{2018}{11}{23}{-1}}
	\task{Moser}{Add new object RevertMove. Started with the first steps for a worker, which simulates games.}
\end{dayentry}

\begin{dayentry}{\DTMdisplaydate{2018}{11}{26}{-1}}
	\task{Moser}{Begin implement a iterativ alpha beta pruning algorithm.}
\end{dayentry}

\begin{dayentry}{\DTMdisplaydate{2018}{11}{28}{-1}}
	\task{Moser}{Uterativ alpha beta pruning algorithm works. Fixing some errors from the cache.}
\end{dayentry}

\begin{dayentry}{\DTMdisplaydate{2018}{11}{29}{-1}}
	\task{Moser}{Tried some opimisations.}
\end{dayentry}

\begin{dayentry}{\DTMdisplaydate{2018}{12}{3}{-1}}
	\task{Moser}{Added some tests. Also added generations to the website.}
\end{dayentry}

\begin{dayentry}{\DTMdisplaydate{2018}{12}{5}{-1}}
	\task{Moser}{Improved some code and added some tests.}
\end{dayentry}

\begin{dayentry}{\DTMdisplaydate{2018}{12}{6}{-1}}
	\task{Moser}{Fixed some small bugs, added tests and cleaned up some code. Implemented GameSimulator and EvolutionController for the web.}
\end{dayentry}

\begin{dayentry}{\DTMdisplaydate{2018}{12}{7}{-1}}
	\task{Moser}{Improved the usability of the website.}
\end{dayentry}

\begin{dayentry}{\DTMdisplaydate{2018}{12}{8}{-1}}
	\task{Jampen}{Added first part of the introduction.}
\end{dayentry}

\begin{dayentry}{\DTMdisplaydate{2018}{12}{10}{-1}}
	\task{Moser}{Improved train and mutate function and added support for modals.}
\end{dayentry}

\begin{dayentry}{\DTMdisplaydate{2018}{12}{12}{-1}}
	\task{Moser}{Added some improvements for stability.}
\end{dayentry}

\begin{dayentry}{\DTMdisplaydate{2018}{12}{14}{-1}}
	\task{Moser}{Improved Code, replaced callbacks with futures and completed the modal for new evolutions. Split up classes of the evolution.}
	\task{Jampen}{Added sections in the documentation.}
\end{dayentry}

\begin{dayentry}{\DTMdisplaydate{2018}{12}{15}{-1}}
	\task{Moser}{Added safe and load function for the trained eras and added a progressbar.}
	\task{Jampen}{Add rules of the game to the documentation.}
\end{dayentry}

\section{Reflexion}
\subsection{Damian Moser}
When Stefan and I brainstormed about the theme of our project, i was glad that we found a project, which interested both of us. Especially that the project will be about self learning algorithms. I personally am very interested in this topic and was fully motivated to do a practical project about this. Even though I haven't done anything similar like this ever before, I had some foreknowledge from articles and videos about self learning algorithms. Therefore we had a concept pretty quick. 

The first steps of the practical part weren't that exciting, but as the program functionality grew, it became more and more fascinating. At the beginning we didn't knew, whether the project will succeed or fail. So it was an even bigger pleasure, when the first generations had done some visible improvements. \\

Writing the document in \LaTeX\ was definitely the right decision. I had some experience from some previous projects i had done at work. So we had a well-tested template, which i knew will work for this project. Additionally this decision allowed us to work with a GIT, a version management tool. We needed this anyway to share our sources of the program. With \LaTeX\ it self we had just a few problem and i'm pretty sure, less problem as we would had with word. GIT was more frustrating. Some times we had problems with the control of it or changes disappeared. Thankfully we never lost any work, but i was sometimes a little bit time-consuming.\\

When I look back at our project, I think we had done some great work. We built a self-learning and more importantly a working program. Also i think the documentation describes the each part of the program well and understandable. At least i hope it does.

\subsection{Stefan Jampen}
We started very early with the basic requirements for the project, such as setting up de GIT repository.
Damian proposed to work with \LaTeX\ for our documentation, cause it's perfectly suitable for version control. \LaTeX\ is often used in scientific works as well. Because of that, the IDPA, which intends to prepare us for later work at universities, was perfect to try LaTeX\ for the first time. I'm glad, we made this decision. Once the project was set up, everything worked smooth and we never had problems because all files were always up to date.\\

At the point we made the decision to make an algorithm, I underestimated the complexity to implement the algorithm in code. I wasn't very practised in Object Oriented Programming (OOP). I understood the basic concepts, but didn't have any practice experience. Because of that, I wasn't qualified to write code that really contributed to the practical implementation.
I had a little experience in static website scripting, and could finally contribute my knowledge when it came to the implementation of the user interface. In retrospective, I gained a lot of knowledge in dynamic website programming and was able to grasp some concepts, especially in OOP, which I struggled with earlier. 
For a next project, I would make sure, that I would be able to contribute more to the core of the project.
Despite that I struggled with coding, I could learn a lot from Damians experience, which I really appreciate.\\

I'm very satisfied with the result of our project. We have a perfectly working algorithm, which is extremely hard to beat. Besides that, we also have a good looking website, which makes our project visible to interested people. All in all, I'm proud of us and the final project.






















